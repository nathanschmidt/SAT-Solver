\section{Introduction}\label{sec:intro}

The \emph{Boolean Satisfiability Problem (SAT)} consists in determining whether, given a formula of propositional logic, there exists a \emph{valuation} that satisfies it, i.e., a mapping from the unknowns of the formula to boolean values such that the formula holds. 
Great numbers of problems in computer science can easily be reduced to SAT, and even though SAT is known to be \textbf{NP}-complete by the Cook-Levin theorem of 1971 \cite{cook1971} and its many proofs since then, there exist many efficient \emph{SAT-Solvers} for certain classes of formulas.

\bigbreak

In this project, we implement a small formally verified SAT-Solver in the \textsc{Coq} proof assistant for formulas containing conjunctions, disjunctions, implications, and negations. 
To that end, we start by formalizing the syntax of such formulas. 
Then, we introduce their semantic interpretation given a specified valuation. 
Following the implementation of a syntactic optimizer that we show to simplify a formula to its minimal form, we conclude this work by implementing the actual solver based on a brute-force search algorithm and prove it to be both correct and complete.